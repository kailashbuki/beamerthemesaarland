% the sample slide is created with 16:9 aspect ratio
\documentclass[aspectratio=169]{beamer}

% remove the options if you do not want to have them
\usetheme[
	background=saarland,
	logo=uds
]{saarland}

% the background and logo are in the images directory
\graphicspath{{images/}}

% information for the title page
\author{Kailash Budhathoki}
\title{Saarland Beamer Theme}
\subtitle{An unofficial theme for Saarland University}
\institute{Max Planck Institute for Informatics and Saarland University}
\date{\today}

\begin{document}
	% use plain option to remove the page number from the title slide
	\begin{frame}[plain]
		\titlepage
	\end{frame}
	
	\begin{frame}{Slide Full of Lists}
	Saarland is a German state in the dynamic border triangle of Germany, France and Luxembourg. 
		\begin{itemize}
			\item Saarland in figures
				\begin{itemize}
					\item \textbf{Federal state since}: 1st January 1957
					\item \textbf{Area}: 2,569.69 $\text{km}^2$
					\item \textbf{Highest mountain}: 695m (Dolberg in Hunsrück)
					\item \textbf{Population}: 995,597 (31st December 2015)
					\item \textbf{Unemployment rate}: 6.5\% (June 2017)
				\end{itemize}
			\item History
				\begin{itemize}
					\item \textbf{1960--1987}: France forms a Saar province as part of its reunification policies
					\item \textbf{13th January 1935}: Referendum reinstates Saarland into the Third Reich
					\item \textbf{1947}: Saarland is annexed to France in economic terms
					\item \textbf{1957}: Saarland becomes the 10th Federal state of the Federal republic of Germany
				\end{itemize}
			\item Credits: \url{https://www.saarland.de/}
		\end{itemize}
	\end{frame}

	\begin{frame}{Blocks}
		\framesubtitle{This is a subtitle}
		\begin{block}{Standard Block}
			This is a standard block.
		\end{block}
		
		\begin{exampleblock}{Example Block}
			This is an example block.
		\end{exampleblock}
		
		\begin{alertblock}{Alert Block}
			This is an alert block.
		\end{alertblock}
	\end{frame}
	
	\begin{frame}{Math}
		Mathematics is the queen of sciences and arithmetic is the queen of mathematics.
		\begin{align*}
			\Pr(Y \geq 120) &= \Pr\left(Y-n\mu \geq 120-n\mu \right)\\
			&= \Pr\left( \frac{Y-n\mu }{\sqrt{n}\sigma} \geq \frac{120-n\mu }{\sqrt{n}\sigma} \right)\\
			&=\Pr\left( Z \geq \frac{120-n\mu }{\sqrt{n}\sigma} \right)\\
			&=\Pr\left( Z \geq \frac{120-100 \cdot 1 }{10 \cdot 1} \right)\\
			&=\Pr\left( Z \geq 2\right)
		\end{align*}
	\end{frame}

	\begin{frame}{Two Columns}
		We can also add two columns in the slides.
		\begin{columns}[t]
			\begin{column}[T]{0.4\textwidth}
				This is the first column. In this column, we can also add a block for instance.
				\vspace{1em}
				\begin{block}{Block}
					I am a block in a column.
				\end{block}
			\end{column}
			\begin{column}[T]{0.4\textwidth}
				\begin{itemize}
					\item In this column,
					\item we just add the
					\item bullet points.
				\end{itemize}
			\end{column}
		\end{columns}
	\end{frame}
	\begin{frame}{Acknowledgements}
		This theme is inspired by Flip theme (creator: Flip Tanedo). The official beamer user guide was also very handy during the development.
	\end{frame}
\end{document}
